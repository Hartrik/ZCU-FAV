\documentclass[12pt, a4paper]{report}

\usepackage[czech]{babel}
\usepackage[IL2]{fontenc}
\usepackage[utf8]{inputenc}
\usepackage{lmodern}  % lepší kvalita PDF

\usepackage[a4paper,top=3cm,bottom=3cm,left=3cm,right=3cm,marginparwidth=1.75cm]{geometry}

\usepackage{graphicx}
\usepackage{titling}
\usepackage{enumitem}
\usepackage{caption}
\usepackage{float}
\usepackage{pdfpages}
\usepackage{verbatim}
\usepackage{amsmath}

\usepackage{listings}
\lstset{numbers=left,
	inputencoding=latin1,
	basicstyle=\scriptsize\ttfamily,
	keywordstyle=\color{blue},
	breaklines=true, 
	showtabs=false,
	showstringspaces=false,
	numberstyle=\tiny\color{gray},
}

\usepackage{multirow}

\usepackage{pkg-custom-commands}
\usepackage{pkg-url}


% údaje na titulní straně
\title{Samostatná práce}
\def \thesubtitle {KIV/PPR}
\author{Patrik Harag}
\def \theauthoremail {harag@students.zcu.cz}
\def \theauthorid {A18N0084P}

\begin{document}

\begin{titlepage}
	\begin{figure}
		\includegraphics[height=50mm]{img-fav-logo}
	\end{figure}
	
	\centering
	{\large \hspace{1mm} \par} % tady musí být nějaký text jinak nefunguje vertikální odsazení
	\vspace{15ex}
	
	{\huge\bfseries \thetitle \par}
	\vspace{2ex}
	{\scshape\Large \thesubtitle \par}
	\vspace{15ex}
	{\Large\itshape \theauthor \par}
	\vspace{2ex}
	{\texttt{\theauthoremail} \par}
	\vspace{1ex}
	{\texttt{\theauthorid} \par}
	
	\vfill

	{\today\par}
\end{titlepage}

\chapter{Zadání}
\paragraph{Standardní zadání}
Implementujte buď stávající, nebo si navrhněte vlastní, evoluční algoritmus dle specifikovaného rozhraní - viz zdrojové soubory, soubor solver.cpp, funkce do\_solve\_generic. Algoritmus implementujte alespoň ve dvou verzích (SMP, OpenCL nebo MPI), čímž získáte alespoň dva solvery. V souboru descriptor.h si vygenerujte nové GUID pro vaše solvery a zadejte jejich název dle uvedeného vzoru. Nic jiného v tomto souboru neměnte.

\chapter{Analýza}
Pro řešení byl zvolen Spiral Optimization Algorithm.

\section{Spiral Optimization Algorithm}
Spiral Optimization Algorithm (SPO) je evoluční algoritmus určený pro hledání minima funkce. Algoritmus byl průvodně popsán \cite{spo-2d} pro použití v 2-rozměrném prostoru, ale v článku \cite{spo-nd} byl upraven i pro použití v n-rozměrném prostoru.

\paragraph{Algoritmus}
\begin{enumerate}
	\item Nastavení parametrů:
	\begin{itemize}
		\item $k_{max}$ = počet iterací,
		\item $m$ = počet bodů prohledávání, $m > 1$,
		\item $\theta$ = úhel rotace, $0 < \theta < 2\pi$,
		\item $r$ = velikost kroku, $0 < r < 1$.
	\end{itemize}
	\item Vygenerování $m$ bodů prohledávání a určení bodu $x^{\star}$ jako bodu s nejmenší funkční hodnotou.
	\item Rotace ostatních bodů kolem bodu $x^{\star}$ o úhel $\theta$ s velikostí kroku $r$.
	\item Určení funkční hodnoty pro všechny body.
	\item Určení bodu $x^{\star}$ jako bodu s nejmenší funkční hodnotou.
	\item Kontrola zastavovací podmínky $k > k_{max}$, jinak zpět ke kroku 3.
	\item Prohlášení $x^{\star}$ za řešení.
\end{enumerate}

%\paragraph{Rotace}


\chapter{Implementace}
Byly implementovány celkem tři solvery.
Obecná problematika bude vysvětlena na prvním z nich. U ostatních budou vysvětleny už jen specifika související s použitím dané technologie.

\section{Solver využívající sekvenční výpočet}
Algoritmus samotný je poměrně jednoduchý a jeho implementace byla přímočará. Za parametry byly zvoleny $r = 0.95$ a $\theta = \pi / 4$.
Za zmínku však stojí problematika \emph{generování startovacích bodů} a \emph{zastavovací podmínka}.

\paragraph{Generování startovacích bodů}
Pro co nejvyšší pravděpodobnost nalezení řešení je vhodné prostor co nejrovnoměrněji pokrýt.
Bylo provedeno mnoho experimentů a~jako nejlepší řešení se ukázala kombinace určených bodů doplněná náhodnými body.
Určenými body se myslí bod uprostřed prohledávaného prostoru a~body v~jeho rozích.
Přičemž se body neumísťují úplně do rohu, ale pouze do jeho blízkosti, relativně podle velikosti prostoru.
Tento parametr je nastaven na 0.1 z celkové šířky dané dimenze.
Například pro 1D prostor (-1, 1) by to byly body 1.0 (uprostřed), -0.8 (jeden roh) a 0.8 (druhý roh).

\paragraph{Zastavovací podmínka}
Zavedení zastavovací podmínky je nejsnazší a nejjednodušší způsob jak urychlit výpočet.
Na duhou stranu špatná podmínka může přivodit předčasné ukončení výpočtu.
Po různých pokusech se jako nejlepší řešení ukázalo sečtení hodnot funkce pro všechny body a kontrola jestli se tato suma zmenšuje.
Pokud se suma určitý počet iterací nezmenší (nastaveno na 256), tak se výpočet ukončí.
Tak je možné ušetřit až 98 \% času.

\section{Solver využívající TBB}
Jelikož je SPO iterativní algoritmus, je možné paralelizovat výpočty pouze v rámci jedné iterace.

\paragraph{Paralelizované části}
\begin{itemize}
	\item Počáteční výpočet funkční hodnoty (část kroku 2 algoritmu SPO). Uveden na Výpisu \ref{ex-tbb}.
	\item Rotace bodu a výpočet funkční hodnoty (krok 3 a 4 algoritmu SPO).
\end{itemize}
\noindent
Na obě části byla použita funkce \code{tbb::parallel\_for} s \code{tbb::blocked\_range}.
Jiná vhodná místa k paralelizaci nebyla nalezena.

\lstinputlisting[label={ex-tbb}, caption={Ukázka použití \code{tbb::parallel\_for}}]{tbbexample.cpp}

\section{Solver využívající OpenCL}
Pro algoritmus SPO není implementace pro grafickou kartu příliš vhodná a solver tak byl vytvořen spíše z výukových důvodů.

\paragraph{Přesnost čísel s plovoucí desetinnou čárkou}
Při implementaci jsem řešil problém, že žádné ze zařízení nepodporovalo \code{double}.
Standardně se podpora pro \code{double} povoluje příkazem: \verb|#pragma OPENCL EXTENSION cl_khr_fp64: enable|, ovšem po\-uze pokud je rozšíření \code{cl\_khr\_fp64} podporované.
V mém případě tomu tak nebylo, viz Výpis \ref{devices}, a tak jsem musel na grafické kartě počítat s hodnotami typu \code{float}.

\lstinputlisting[label={devices}, caption={Informace o dostupných zařízeních}]{devices.txt}

\paragraph{Paralelizované části}
Paralelizováno bylo maticové násobení použité při výpočtu rotace (část kroku 3 algoritmu SPO).
Původně bylo paralelizováno také odčítání vektorů, ale později byla tato část z výkonnostních důvodů odebrána.
Jako zajímavost však byl tento kód ponechán v příloze.

\chapter{Výsledky}
Testy probíhaly na notebooku s procesorem Intel Core i3-4000M (2.4 GHz, 2 jádra) a 8 GB RAM, který je vybaven integrovaným grafickým čipem.

\section{Přesnost}
Tabulky \ref{tbl:results3}, \ref{tbl:results8} a \ref{tbl:results40} ukazují rozdíl nalezeného optima od optima dané funkce pro každý solver, pro dimenze 3, 8, a 40.
Výsledek je vždy stejný pro každý běh.

\section{Rychlost}
Tabulky \ref{tbl:time3}, \ref{tbl:time8} a \ref{tbl:time40} ukazují průměrnou dobu výpočtu pro dimenze 3, 8, a 40.
Výpočet byl proveden 3x a byla určena i směrodatná odchylka.

\section{Urychlení}
Tabulka \ref{tbl:speedup} porovnává časy běhu sériového a paralelního solveru. Čas běhu je součtem průměrných časů běhu pro všechny problémy přes všechny měřené velikosti populace.

\begin{table}
	\scriptsize
	\caption{Porovnání časů běhu}
	\centering
	\label{tbl:speedup}
	\begin{tabular}{l|l|l|l}
		Dimenze & Doba výpočtu Serial v s & Doba výpočtu TBB v s & Urychlení \\
		\hline
		\hline
		3 & 0.20 & 0.34 & 1.7 \\
		8 & 2.64 & 4.18 & 1.58 \\
		20 & 8.75 & 8.18 & 0.93 \\
		40 & 16.48 & 10.05 & 0.61 \\
	\end{tabular}
\end{table}

\begin{table}
	\scriptsize
	\caption{Přesnost solverů na dané úloze při dimenzi 3}
	\centering
	\label{tbl:results3}
	\begin{tabular}{l|l|l|l|l|l|l|l|l}
		\multirow{2}{*}{Solver} & \multicolumn{8}{c}{Chyba fitness funkce} \\
		& Sphere & Rosenbrock & AbsSum & DeJong4 & Rastrigin & Schwefel & Griewank & Masters \\
		\hline
		\hline
		Serial\_7 & 0.00e+00 & 1.82e-01 & 2.22e-16 & 3.45e-32 & 2.51e-08 & 8.14e-10 & 2.22e-02 & 0.00e+00 \\
		Serial\_15 & 0.00e+00 & 5.34e-01 & 0.00e+00 & 1.05e-10 & 9.95e-01 & 8.14e-10 & 9.86e-03 & 0.00e+00 \\
		Serial\_25 & 0.00e+00 & 7.51e-02 & 3.33e-16 & 1.52e-64 & 9.95e-01 & 8.14e-10 & 5.18e-02 & 0.00e+00 \\
		Serial\_40 & 0.00e+00 & 3.46e-01 & 2.20e-05 & 1.42e-36 & 1.99e+00 & 8.14e-10 & 9.12e-02 & 0.00e+00 \\
		Serial\_60 & 0.00e+00 & 8.44e-01 & 0.00e+00 & 7.29e-63 & 9.95e-01 & 8.14e-10 & 2.22e-02 & 0.00e+00 \\
		Serial\_100 & 0.00e+00 & 2.14e-01 & 1.11e-16 & 0.00e+00 & 0.00e+00 & 8.14e-10 & 2.96e-02 & 0.00e+00 \\
		\hline
		SMP\_7 & 0.00e+00 & 1.82e-01 & 2.22e-16 & 3.45e-32 & 2.51e-08 & 8.14e-10 & 2.22e-02 & 0.00e+00 \\
		SMP\_15 & 0.00e+00 & 5.34e-01 & 0.00e+00 & 1.05e-10 & 9.95e-01 & 8.14e-10 & 9.86e-03 & 0.00e+00 \\
		SMP\_25 & 0.00e+00 & 7.51e-02 & 3.33e-16 & 1.52e-64 & 9.95e-01 & 8.14e-10 & 5.18e-02 & 0.00e+00 \\
		SMP\_40 & 0.00e+00 & 3.46e-01 & 2.20e-05 & 1.42e-36 & 1.99e+00 & 8.14e-10 & 9.12e-02 & 0.00e+00 \\
		SMP\_60 & 0.00e+00 & 8.44e-01 & 0.00e+00 & 7.29e-63 & 9.95e-01 & 8.14e-10 & 2.22e-02 & 0.00e+00 \\
		SMP\_100 & 0.00e+00 & 2.14e-01 & 1.11e-16 & 0.00e+00 & 0.00e+00 & 8.14e-10 & 2.96e-02 & 0.00e+00 \\
		\hline
		OpenCL\_7 & 0.00e+00 & 1.82e-01 & 1.19e-07 & 4.04e-28 & 1.89e-09 & 1.08e-08 & 2.22e-02 & 0.00e+00 \\
		OpenCL\_15 & 0.00e+00 & 5.34e-01 & 1.19e-06 & 1.06e-10 & 9.95e-01 & 1.08e-08 & 9.86e-03 & 0.00e+00 \\
		OpenCL\_25 & 0.00e+00 & 7.51e-02 & 3.58e-07 & 1.01e-26 & 9.95e-01 & 1.46e-08 & 5.18e-02 & 0.00e+00 \\
		OpenCL\_40 & 0.00e+00 & 3.46e-01 & 2.96e-05 & 7.88e-25 & 1.99e+00 & 1.31e-09 & 9.12e-02 & 0.00e+00 \\
		OpenCL\_60 & 0.00e+00 & 8.44e-01 & 2.38e-07 & 4.04e-28 & 9.95e-01 & 8.09e-10 & 2.22e-02 & 0.00e+00 \\
		OpenCL\_100 & 0.00e+00 & 2.14e-01 & 4.17e-07 & 8.08e-28 & 1.05e-09 & 3.81e-09 & 2.96e-02 & 0.00e+00 \\
	\end{tabular}
\end{table}

\begin{table}
	\scriptsize
	\caption{Přesnost solverů na dané úloze při dimenzi 8}
	\centering
	\label{tbl:results8}
	\begin{tabular}{l|l|l|l|l|l|l|l|l}
		\multirow{2}{*}{Solver} & \multicolumn{8}{c}{Chyba fitness funkce} \\
		& Sphere & Rosenbrock & AbsSum & DeJong4 & Rastrigin & Schwefel & Griewank & Masters \\
		\hline
		\hline
		Serial\_7 & 0.00e+00 & 1.88e-02 & 6.66e-16 & 1.25e-61 & 0.00e+00 & 2.17e-09 & 2.02e-01 & 0.00e+00 \\
		Serial\_15 & 0.00e+00 & 7.99e-03 & 1.11e-16 & 1.51e-61 & 0.00e+00 & 2.17e-09 & 2.02e-01 & 0.00e+00 \\
		Serial\_25 & 0.00e+00 & 9.86e-01 & 4.44e-16 & 1.47e-62 & 0.00e+00 & 2.17e-09 & 1.60e-01 & 0.00e+00 \\
		Serial\_40 & 0.00e+00 & 9.47e-01 & 3.33e-16 & 5.77e-63 & 0.00e+00 & 2.17e-09 & 1.16e-01 & 0.00e+00 \\
		Serial\_60 & 0.00e+00 & 8.58e-04 & 2.22e-16 & 1.22e-62 & 0.00e+00 & 2.17e-09 & 5.05e-01 & 0.00e+00 \\
		Serial\_100 & 0.00e+00 & 1.17e-02 & 3.33e-16 & 9.72e-63 & 0.00e+00 & 2.17e-09 & 1.97e-01 & 0.00e+00 \\
		\hline
		SMP\_7 & 0.00e+00 & 1.88e-02 & 6.66e-16 & 1.25e-61 & 0.00e+00 & 2.17e-09 & 2.02e-01 & 0.00e+00 \\
		SMP\_15 & 0.00e+00 & 7.99e-03 & 1.11e-16 & 1.51e-61 & 0.00e+00 & 2.17e-09 & 2.02e-01 & 0.00e+00 \\
		SMP\_25 & 0.00e+00 & 9.86e-01 & 4.44e-16 & 1.47e-62 & 0.00e+00 & 2.17e-09 & 1.60e-01 & 0.00e+00 \\
		SMP\_40 & 0.00e+00 & 9.47e-01 & 3.33e-16 & 5.77e-63 & 0.00e+00 & 2.17e-09 & 1.16e-01 & 0.00e+00 \\
		SMP\_60 & 0.00e+00 & 8.58e-04 & 2.22e-16 & 1.22e-62 & 0.00e+00 & 2.17e-09 & 5.05e-01 & 0.00e+00 \\
		SMP\_100 & 0.00e+00 & 1.17e-02 & 3.33e-16 & 9.72e-63 & 0.00e+00 & 2.17e-09 & 1.97e-01 & 0.00e+00 \\
		\hline
		OpenCL\_7 & 0.00e+00 & 1.86e-02 & 2.98e-07 & 1.92e-26 & 1.13e-10 & 1.22e-09 & 2.02e-01 & 0.00e+00 \\
		OpenCL\_15 & 0.00e+00 & 7.74e-03 & 2.98e-07 & 3.76e-27 & 3.27e-10 & 1.92e-09 & 2.02e-01 & 0.00e+00 \\
		OpenCL\_25 & 0.00e+00 & 9.42e-01 & 2.38e-07 & 8.84e-28 & 1.47e-10 & 1.72e-09 & 1.60e-01 & 0.00e+00 \\
		OpenCL\_40 & 0.00e+00 & 9.02e-01 & 2.38e-07 & 1.16e-26 & 5.64e-11 & 1.92e-09 & 1.16e-01 & 0.00e+00 \\
		OpenCL\_60 & 0.00e+00 & 1.75e-04 & 8.94e-07 & 5.74e-27 & 1.35e-10 & 1.46e-11 & 5.05e-01 & 0.00e+00 \\
		OpenCL\_100 & 0.00e+00 & 1.02e-02 & 5.96e-08 & 1.21e-27 & 1.13e-11 & 1.40e-09 & 1.97e-01 & 0.00e+00 \\
	\end{tabular}
\end{table}

\begin{table}
	\scriptsize
	\caption{Přesnost solverů na dané úloze při dimenzi 40}
	\centering
	\label{tbl:results40}
	\begin{tabular}{l|l|l|l|l|l|l|l|l}
		\multirow{2}{*}{Solver} & \multicolumn{8}{c}{Chyba fitness funkce} \\
		& Sphere & Rosenbrock & AbsSum & DeJong4 & Rastrigin & Schwefel & Griewank & Masters \\
		\hline
		\hline
		Serial\_7 & 0.00e+00 & 2.07e+02 & 3.60e+00 & 1.42e-06 & 3.88e+00 & 1.14e+02 & 3.99e+02 & 0.00e+00 \\
		Serial\_15 & 0.00e+00 & 1.54e+02 & 3.35e+00 & 1.42e-06 & 2.02e+00 & 5.45e+01 & 3.91e+02 & 0.00e+00 \\
		Serial\_25 & 0.00e+00 & 2.28e+02 & 3.31e+00 & 7.22e-07 & 1.57e+00 & 6.84e+01 & 2.72e+02 & 0.00e+00 \\
		Serial\_40 & 0.00e+00 & 6.05e+01 & 2.71e+00 & 4.77e-07 & 5.14e-01 & 1.97e+01 & 2.56e+02 & 0.00e+00 \\
		Serial\_60 & 0.00e+00 & 6.70e+01 & 2.20e+00 & 2.27e-07 & 1.87e-01 & 8.03e+00 & 3.14e+02 & 0.00e+00 \\
		Serial\_100 & 0.00e+00 & 1.03e+02 & 3.21e+00 & 4.11e-07 & 2.74e-01 & 1.01e+01 & 9.51e+01 & 0.00e+00 \\
		\hline
		SMP\_7 & 0.00e+00 & 2.07e+02 & 3.60e+00 & 1.42e-06 & 3.88e+00 & 1.14e+02 & 3.99e+02 & 0.00e+00 \\
		SMP\_15 & 0.00e+00 & 1.54e+02 & 3.35e+00 & 1.42e-06 & 2.02e+00 & 5.45e+01 & 3.91e+02 & 0.00e+00 \\
		SMP\_25 & 0.00e+00 & 2.28e+02 & 3.31e+00 & 7.22e-07 & 1.57e+00 & 6.84e+01 & 2.72e+02 & 0.00e+00 \\
		SMP\_40 & 0.00e+00 & 6.05e+01 & 2.71e+00 & 4.77e-07 & 5.14e-01 & 1.97e+01 & 2.56e+02 & 0.00e+00 \\
		SMP\_60 & 0.00e+00 & 6.70e+01 & 2.20e+00 & 2.27e-07 & 1.87e-01 & 8.03e+00 & 3.14e+02 & 0.00e+00 \\
		SMP\_100 & 0.00e+00 & 1.03e+02 & 3.21e+00 & 4.11e-07 & 2.74e-01 & 1.01e+01 & 9.51e+01 & 0.00e+00 \\
		\hline
		OpenCL\_7 & 0.00e+00 & 2.06e+02 & 3.60e+00 & 1.02e-06 & 3.67e+00 & 1.09e+02 & 3.99e+02 & 0.00e+00 \\
		OpenCL\_15 & 0.00e+00 & 1.54e+02 & 3.35e+00 & 1.22e-06 & 1.99e+00 & 5.08e+01 & 3.91e+02 & 0.00e+00 \\
		OpenCL\_25 & 0.00e+00 & 2.28e+02 & 3.31e+00 & 6.30e-07 & 1.18e+00 & 4.73e+01 & 2.72e+02 & 0.00e+00 \\
		OpenCL\_40 & 0.00e+00 & 6.04e+01 & 2.71e+00 & 3.89e-07 & 4.40e-01 & 1.79e+01 & 2.56e+02 & 0.00e+00 \\
		OpenCL\_60 & 0.00e+00 & 6.70e+01 & 2.20e+00 & 2.16e-07 & 1.65e-01 & 8.11e+00 & 3.14e+02 & 0.00e+00 \\
		OpenCL\_100 & 0.00e+00 & 1.03e+02 & 3.21e+00 & 3.92e-07 & 2.47e-01 & 1.01e+01 & 9.51e+01 & 0.00e+00 \\
	\end{tabular}
\end{table}

\begin{table}
	\scriptsize
	\caption{Průměrná doba výpočtu při dimenzi 3 (první 4 problémy)}
	\centering
	\label{tbl:time3}
	\begin{tabular}{l|l|l|l|l}
		\multirow{2}{*}{Solver} & \multicolumn{4}{c}{Doba výpočtu v sekundách} \\
		& Sphere & Rosenbrock & AbsSum & DeJong4 \\
		\hline
		\hline
		Serial\_7 & 3.07e-04 $\pm$ 9.03e-06 & 3.33e-04 $\pm$ 3.48e-06 & 3.12e-04 $\pm$ 6.06e-06 & 1.39e-03 $\pm$ 6.78e-06 \\
		Serial\_15 & 6.67e-04 $\pm$ 4.18e-06 & 7.54e-04 $\pm$ 5.53e-06 & 6.88e-04 $\pm$ 4.45e-06 & 3.85e-03 $\pm$ 5.35e-06 \\
		Serial\_25 & 1.21e-03 $\pm$ 8.55e-05 & 1.26e-03 $\pm$ 8.37e-06 & 1.15e-03 $\pm$ 1.58e-05 & 5.46e-03 $\pm$ 1.42e-05 \\
		Serial\_40 & 2.01e-03 $\pm$ 3.58e-04 & 2.04e-03 $\pm$ 1.35e-05 & 1.88e-03 $\pm$ 1.19e-05 & 9.03e-03 $\pm$ 1.59e-04 \\
		Serial\_60 & 3.24e-03 $\pm$ 5.42e-04 & 3.11e-03 $\pm$ 1.69e-05 & 2.81e-03 $\pm$ 1.46e-05 & 1.38e-02 $\pm$ 2.78e-05 \\
		Serial\_100 & 4.56e-03 $\pm$ 3.37e-05 & 5.23e-03 $\pm$ 2.61e-04 & 4.53e-03 $\pm$ 2.06e-05 & 2.26e-02 $\pm$ 2.40e-05 \\
		\hline
		SMP\_7 & 3.25e-03 $\pm$ 5.12e-04 & 2.93e-03 $\pm$ 2.50e-05 & 2.92e-03 $\pm$ 2.43e-05 & 3.47e-03 $\pm$ 3.30e-05 \\
		SMP\_15 & 4.38e-03 $\pm$ 5.68e-05 & 4.45e-03 $\pm$ 1.48e-05 & 4.38e-03 $\pm$ 3.69e-05 & 6.46e-03 $\pm$ 3.60e-05 \\
		SMP\_25 & 6.32e-03 $\pm$ 7.66e-04 & 6.14e-03 $\pm$ 1.83e-04 & 5.83e-03 $\pm$ 1.41e-05 & 7.42e-03 $\pm$ 4.45e-05 \\
		SMP\_40 & 8.32e-03 $\pm$ 1.18e-03 & 7.76e-03 $\pm$ 4.27e-05 & 7.64e-03 $\pm$ 8.43e-05 & 1.01e-02 $\pm$ 4.55e-05 \\
		SMP\_60 & 9.78e-03 $\pm$ 1.00e-03 & 9.39e-03 $\pm$ 1.89e-05 & 9.08e-03 $\pm$ 2.57e-05 & 1.28e-02 $\pm$ 1.19e-04 \\
		SMP\_100 & 1.28e-02 $\pm$ 2.76e-04 & 1.28e-02 $\pm$ 8.09e-05 & 1.22e-02 $\pm$ 6.46e-05 & 1.85e-02 $\pm$ 2.25e-04 \\
		\hline
		OpenCL\_7 & 1.79e+00 $\pm$ 1.69e-01 & 1.69e+00 $\pm$ 2.07e-02 & 1.70e+00 $\pm$ 2.77e-02 & 1.61e+00 $\pm$ 1.37e-02 \\
		OpenCL\_15 & 3.76e+00 $\pm$ 7.53e-01 & 3.15e+00 $\pm$ 3.06e-01 & 3.88e+00 $\pm$ 3.42e-02 & 2.95e+00 $\pm$ 2.43e-02 \\
		OpenCL\_25 & 5.51e+00 $\pm$ 1.32e+00 & 5.15e+00 $\pm$ 3.91e-02 & 4.64e+00 $\pm$ 7.76e-02 & 5.61e+00 $\pm$ 1.33e+00 \\
		OpenCL\_40 & 7.44e+00 $\pm$ 3.79e-02 & 7.83e+00 $\pm$ 4.58e-02 & 7.78e+00 $\pm$ 1.43e-01 & 7.38e+00 $\pm$ 3.28e-02 \\
		OpenCL\_60 & 1.14e+01 $\pm$ 4.66e-01 & 1.14e+01 $\pm$ 2.64e-02 & 1.10e+01 $\pm$ 1.31e+00 & 1.12e+01 $\pm$ 1.36e+00 \\
		OpenCL\_100 & 1.83e+01 $\pm$ 9.47e-01 & 1.83e+01 $\pm$ 1.63e-01 & 1.90e+01 $\pm$ 6.83e-02 & 1.80e+01 $\pm$ 4.47e-02 \\
	\end{tabular}
\end{table}

\begin{table}
	\scriptsize
	\caption{Průměrná doba výpočtu při dimenzi 8 (první 4 problémy)}
	\centering
	\label{tbl:time8}
	\begin{tabular}{l|l|l|l|l}
		\multirow{2}{*}{Solver} & \multicolumn{4}{c}{Doba výpočtu v sekundách} \\
		& Sphere & Rosenbrock & AbsSum & DeJong4 \\
		\hline
		\hline
		Serial\_7 & 3.19e-04 $\pm$ 9.78e-05 & 5.39e-03 $\pm$ 7.95e-05 & 8.58e-04 $\pm$ 7.82e-06 & 3.76e-03 $\pm$ 1.58e-05 \\
		Serial\_15 & 6.57e-04 $\pm$ 1.26e-04 & 3.28e-02 $\pm$ 5.14e-04 & 2.15e-03 $\pm$ 5.57e-06 & 9.68e-03 $\pm$ 8.12e-06 \\
		Serial\_25 & 9.37e-04 $\pm$ 1.19e-04 & 3.26e-01 $\pm$ 4.81e-03 & 3.09e-03 $\pm$ 1.01e-05 & 1.43e-02 $\pm$ 2.44e-05 \\
		Serial\_40 & 1.73e-03 $\pm$ 4.72e-04 & 5.16e-01 $\pm$ 1.17e-03 & 5.90e-03 $\pm$ 1.74e-05 & 2.31e-02 $\pm$ 1.69e-05 \\
		Serial\_60 & 2.23e-03 $\pm$ 6.88e-04 & 1.29e-01 $\pm$ 3.62e-04 & 8.65e-03 $\pm$ 3.26e-05 & 4.13e-02 $\pm$ 3.20e-04 \\
		Serial\_100 & 5.23e-03 $\pm$ 5.55e-04 & 1.15e+00 $\pm$ 4.44e-02 & 1.77e-02 $\pm$ 1.34e-05 & 5.56e-02 $\pm$ 1.38e-04 \\
		\hline
		SMP\_7 & 1.53e-03 $\pm$ 3.85e-04 & 2.27e-02 $\pm$ 7.86e-05 & 3.80e-03 $\pm$ 9.14e-06 & 5.26e-03 $\pm$ 5.01e-05 \\
		SMP\_15 & 1.78e-03 $\pm$ 7.04e-05 & 9.38e-02 $\pm$ 4.30e-04 & 6.68e-03 $\pm$ 3.20e-05 & 9.57e-03 $\pm$ 3.05e-05 \\
		SMP\_25 & 2.52e-03 $\pm$ 8.66e-05 & 7.78e-01 $\pm$ 1.66e-03 & 7.92e-03 $\pm$ 2.26e-05 & 1.18e-02 $\pm$ 2.01e-05 \\
		SMP\_40 & 3.34e-03 $\pm$ 1.02e-04 & 1.03e+00 $\pm$ 6.16e-03 & 1.25e-02 $\pm$ 7.05e-05 & 1.66e-02 $\pm$ 2.67e-04 \\
		SMP\_60 & 5.66e-03 $\pm$ 1.79e-03 & 2.12e-01 $\pm$ 3.82e-04 & 1.51e-02 $\pm$ 1.05e-04 & 2.71e-02 $\pm$ 4.83e-05 \\
		SMP\_100 & 7.08e-03 $\pm$ 2.29e-03 & 1.61e+00 $\pm$ 2.90e-03 & 2.64e-02 $\pm$ 7.26e-05 & 3.22e-02 $\pm$ 1.02e-04 \\
		\hline
		OpenCL\_7 & 1.03e+00 $\pm$ 1.60e-01 & 1.89e+01 $\pm$ 1.23e-01 & 1.92e+00 $\pm$ 9.16e-03 & 2.07e+00 $\pm$ 2.39e-02 \\
		OpenCL\_15 & 1.53e+00 $\pm$ 9.42e-03 & 5.38e+01 $\pm$ 1.52e+00 & 3.92e+00 $\pm$ 3.34e-02 & 4.59e+00 $\pm$ 1.37e+00 \\
		OpenCL\_25 & 2.33e+00 $\pm$ 3.58e-02 & 8.12e+02 $\pm$ 1.12e+00 & 5.75e+00 $\pm$ 7.51e-02 & 5.63e+00 $\pm$ 2.78e-02 \\
		OpenCL\_40 & 3.54e+00 $\pm$ 1.88e-02 & 1.27e+03 $\pm$ 6.37e+00 & 8.34e+00 $\pm$ 5.60e-02 & 7.60e+00 $\pm$ 3.89e-02 \\
		OpenCL\_60 & 5.10e+00 $\pm$ 3.59e-02 & 1.78e+03 $\pm$ 2.47e+00 & 1.17e+01 $\pm$ 6.22e-02 & 1.14e+01 $\pm$ 9.17e-02 \\
		OpenCL\_100 & 8.22e+00 $\pm$ 5.26e-02 & 1.20e+03 $\pm$ 7.17e+00 & 2.84e+01 $\pm$ 8.94e-02 & 2.24e+01 $\pm$ 1.18e+00 \\
	\end{tabular}
\end{table}

\begin{table}
	\scriptsize
	\caption{Průměrná doba výpočtu při dimenzi 40 (první 4 problémy)}
	\centering
	\label{tbl:time40}
	\begin{tabular}{l|l|l|l|l}
		\multirow{2}{*}{Solver} & \multicolumn{4}{c}{Doba výpočtu v sekundách} \\
		& Sphere & Rosenbrock & AbsSum & DeJong4 \\
		\hline
		\hline
		Serial\_7 & 9.29e-02 $\pm$ 1.90e-02 & 2.74e-01 $\pm$ 7.07e-04 & 8.30e-02 $\pm$ 1.65e-04 & 2.01e-01 $\pm$ 3.45e-04 \\
		Serial\_15 & 8.75e-02 $\pm$ 1.62e-05 & 9.12e-01 $\pm$ 3.53e-03 & 8.77e-02 $\pm$ 2.85e-04 & 5.14e-01 $\pm$ 2.69e-03 \\
		Serial\_25 & 9.32e-02 $\pm$ 1.58e-04 & 9.37e-02 $\pm$ 6.16e-04 & 9.35e-02 $\pm$ 5.96e-04 & 7.33e-01 $\pm$ 4.93e-04 \\
		Serial\_40 & 1.02e-01 $\pm$ 1.19e-04 & 1.81e-01 $\pm$ 1.22e-03 & 1.03e-01 $\pm$ 6.46e-04 & 6.08e+00 $\pm$ 2.38e-02 \\
		Serial\_60 & 1.15e-01 $\pm$ 9.36e-05 & 1.16e-01 $\pm$ 1.77e-04 & 1.20e-01 $\pm$ 6.10e-03 & 7.20e-01 $\pm$ 7.65e-02 \\
		Serial\_100 & 1.36e-01 $\pm$ 3.09e-04 & 1.38e-01 $\pm$ 2.84e-04 & 1.37e-01 $\pm$ 3.48e-04 & 6.14e-01 $\pm$ 4.95e-03 \\
		\hline
		SMP\_7 & 8.28e-02 $\pm$ 5.84e-04 & 2.56e-01 $\pm$ 1.07e-03 & 8.28e-02 $\pm$ 3.11e-04 & 1.66e-01 $\pm$ 5.95e-04 \\
		SMP\_15 & 8.49e-02 $\pm$ 9.90e-06 & 6.49e-01 $\pm$ 2.16e-03 & 8.52e-02 $\pm$ 3.02e-04 & 3.17e-01 $\pm$ 9.21e-04 \\
		SMP\_25 & 8.78e-02 $\pm$ 2.69e-04 & 8.81e-02 $\pm$ 4.80e-04 & 8.80e-02 $\pm$ 4.45e-04 & 3.94e-01 $\pm$ 2.20e-03 \\
		SMP\_40 & 9.31e-02 $\pm$ 3.90e-04 & 1.36e-01 $\pm$ 1.93e-04 & 9.30e-02 $\pm$ 2.82e-04 & 2.87e+00 $\pm$ 1.34e-03 \\
		SMP\_60 & 9.92e-02 $\pm$ 1.57e-04 & 1.00e-01 $\pm$ 2.06e-04 & 9.95e-02 $\pm$ 1.30e-05 & 3.39e-01 $\pm$ 4.77e-04 \\
		SMP\_100 & 1.09e-01 $\pm$ 3.04e-04 & 1.10e-01 $\pm$ 3.14e-04 & 1.09e-01 $\pm$ 3.39e-04 & 3.10e-01 $\pm$ 7.46e-04 \\
		\hline
		OpenCL\_7 & 1.14e+00 $\pm$ 1.92e-01 & 2.21e+01 $\pm$ 1.33e+00 & 1.04e+00 $\pm$ 1.22e-02 & 3.29e+01 $\pm$ 1.37e+00 \\
		OpenCL\_15 & 1.70e+00 $\pm$ 2.52e-02 & 3.16e+01 $\pm$ 4.84e-02 & 1.69e+00 $\pm$ 1.35e-02 & 3.84e+01 $\pm$ 2.38e+00 \\
		OpenCL\_25 & 2.53e+00 $\pm$ 2.76e-02 & 2.57e+00 $\pm$ 4.45e-02 & 2.57e+00 $\pm$ 3.44e-02 & 3.78e+01 $\pm$ 1.34e+00 \\
		OpenCL\_40 & 3.70e+00 $\pm$ 5.25e-02 & 2.29e+01 $\pm$ 1.77e-01 & 3.80e+00 $\pm$ 7.72e-02 & 9.63e+01 $\pm$ 4.52e-01 \\
		OpenCL\_60 & 5.27e+00 $\pm$ 5.97e-02 & 5.42e+00 $\pm$ 6.83e-02 & 5.47e+00 $\pm$ 6.97e-02 & 3.39e+01 $\pm$ 1.44e+00 \\
		OpenCL\_100 & 9.23e+00 $\pm$ 1.25e+00 & 8.65e+00 $\pm$ 6.83e-02 & 8.76e+00 $\pm$ 4.76e-02 & 6.18e+01 $\pm$ 1.41e+00 \\
	\end{tabular}
\end{table}

\chapter{Závěr}
Implementovaný algoritmus zafungoval na některé problémy velmi dobře a na jiné hůře. U některých problémů bylo nalezeno optimum.

Sekvenční solver a SMP solver dávají vždy stejné výsledky. Výsledky OpenCL solveru se někdy liší kvůli převodu na \code{float}.
Výsledek závisí především na vygenerovaných startovacích bodech, jelikož algoritmus nemá prostředky, jak by zabránil uváznutí v lokálním minimu.

Pro algoritmus SPO není implementace pro grafickou kartu příliš vhodná a~solver pro OpenCL tak byl vytvořen spíše z výukových důvodů.
K horšímu výkonu přispěl také fakt, že se čísla musela převádět z typu \code{double} na typ \code{float}.

Pro menší počet dimenzí je efektivnější sériový solver. S rostoucí dimenzí se začíná paralelizace TBB solveru vyplácet.
Práci by vylepšilo měření na stroji s více jádry, aby bylo více patrné urychlení TBB solveru vůči sériovému solveru.

\bibliographystyle{plain}
\bibliography{references}

\chapter*{Přílohy}
\section*{Odčítání vektorů v prostředí OpenCL}
\lstinputlisting[]{openclexample.cpp}

\end{document}
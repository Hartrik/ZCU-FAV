\documentclass[12pt, a4paper]{article}

\usepackage[czech]{babel}
\usepackage[IL2]{fontenc}
\usepackage[utf8]{inputenc}
\usepackage{lmodern}  % lepší kvalita PDF

\usepackage[a4paper,top=3cm,bottom=3cm,left=3cm,right=3cm,marginparwidth=1.75cm]{geometry}

\usepackage{graphicx}
\usepackage{titling}
\usepackage{enumitem}
\usepackage{caption}
\usepackage{float}
\usepackage{pdfpages}
\usepackage{verbatim}
\usepackage{amsmath}

\usepackage{pkg-custom-commands}
\usepackage{pkg-url}

% údaje na titulní straně
\title{Cvičení 1}
\def \thesubtitle {KIV/VSS}
\author{Patrik Harag}
\def \theauthoremail {harag@students.zcu.cz}
\def \theauthorid {A18N0084P, nar. 10. května}

\begin{document}

\begin{titlepage}
	\begin{figure}
		\includegraphics[height=50mm]{img-fav-logo}
	\end{figure}
	
	\centering
	{\large \hspace{1mm} \par} % tady musí být nějaký text jinak nefunguje vertikální odsazení
	\vspace{15ex}
	
	{\huge\bfseries \thetitle \par}
	\vspace{2ex}
	{\scshape\Large \thesubtitle \par}
	\vspace{15ex}
	{\Large\itshape \theauthor \par}
	\vspace{2ex}
	{\texttt{\theauthoremail} \par}
	\vspace{1ex}
	{\texttt{\theauthorid} \par}
	
	\vfill

	{\today\par}
\end{titlepage}

\section{Zadání}

\paragraph{5. Lichoběžníkové rozdělení}
\begin{itemize}
	\item $f(x) = 0$ pro $x < a$
	\item $f(x)$ lin. roste pro $b > x > a$
	\item $f(x)$ je konst pro $c > x > b$
	\item $f(x) = 0$ pro $x > c$
\end{itemize}

\section{Řešení}
Byl vytvořen program v jazyce Groovy generující zadané rozdělení.

\paragraph{Generování čísel}
Pro mapování náhodných čísel z~unoformního generátoru\\ (\code{java.util.Random}) využívá vylučovací metodu.

\paragraph{Statistiky}
Program zjištuje střední hodnotu a rozptyl vygenerovaných čísel. V~compile time je možné zvolit mezi přesným výpočtem (vyžaduje uložení všech čísel) a~přibližným výpočtem (při výpočtu rozptylu se využije aktuální střední hodnotu z~dosud vygenerovaných čísel). Jako výchozí je nastaven přibližný výpočet.

\paragraph{Teoretické statistiky}
Pro výpočet teoretické střední hodnoty byl použit známý vzorec\footnote{\url{https://en.wikipedia.org/wiki/Trapezoidal\_distribution}} obdobného lichoběžníkového rozdělení, ve kterém $d = c + e$, kde $e$ je kladné číslo blízké nule (aby nedošlo k dělení nulou).
\begin{equation}
	\frac{1}{3(d+c-b-a)}\left(\frac{d^3-c^3}{d-c}-\frac{b^3-a^3}{b-a}\right)
\end{equation}

\paragraph{Kompilace a spuštění}
Program se zkompiluje spuštěním \code{build.bat} (je nutné mít nainstalovaný gradle) a spustí spuštěním \code{run.bat} (vyžaduje Javu 8).
Program lze také spustit s parametry \lt\emph{počet vzorků}, \emph{a}, \emph{b}, \emph{c}\gt \space v uvedeném pořadí.
Přípustné je pouze spuštění bez parametrů nebo se všemi parametry.

\end{document}
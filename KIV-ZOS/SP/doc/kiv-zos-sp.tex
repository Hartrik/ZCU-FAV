\documentclass[12pt, a4paper]{report}

\usepackage[czech]{babel}
\usepackage[IL2]{fontenc}
\usepackage[utf8]{inputenc}
\usepackage{lmodern}  % lepší kvalita PDF

\usepackage[a4paper,top=3cm,bottom=3cm,left=3cm,right=3cm,marginparwidth=1.75cm]{geometry}

\usepackage{amsmath}
\usepackage{graphicx}
\usepackage{titling}
\usepackage{enumitem}
\usepackage[colorlinks=true, allcolors=black]{hyperref}
\usepackage{url}
\usepackage{caption}
\usepackage{float}

\usepackage{pdfpages}


% formátování zdrojového kódu
\usepackage{listings}
\usepackage{color}

\definecolor{dkgreen}{rgb}{0,0.6,0}
\definecolor{gray}{rgb}{0.5,0.5,0.5}
\definecolor{mauve}{rgb}{0.58,0,0.82}

\lstset{frame=tb,
	language=C,
	aboveskip=3mm,
	belowskip=3mm,
	showstringspaces=false,
	columns=flexible,
	basicstyle={\small\ttfamily},
	numbers=none,
	numberstyle=\tiny\color{gray},
	keywordstyle=\color{blue},
	commentstyle=\color{dkgreen},
	stringstyle=\color{mauve},
	breaklines=true,
	breakatwhitespace=true,
	tabsize=4,
	extendedchars=true,
	literate=%
	{á}{{\'a}}1
	{č}{{\v{c}}}1
	{ď}{{\v{d}}}1
	{é}{{\'e}}1
	{ě}{{\v{e}}}1
	{í}{{\'i}}1
	{ň}{{\v{n}}}1
	{ó}{{\'o}}1
	{ř}{{\v{r}}}1
	{š}{{\v{s}}}1
	{ť}{{\v{t}}}1
	{ú}{{\'u}}1
	{ů}{{\r{u}}}1
	{ý}{{\'y}}1
	{ž}{{\v{z}}}1
	{Á}{{\'A}}1
	{Č}{{\v{C}}}1
	{Ď}{{\v{D}}}1
	{É}{{\'E}}1
	{Ě}{{\v{E}}}1
	{Í}{{\'I}}1
	{Ň}{{\v{N}}}1
	{Ó}{{\'O}}1
	{Ř}{{\v{R}}}1
	{Š}{{\v{S}}}1
	{Ť}{{\v{T}}}1
	{Ú}{{\'U}}1
	{Ů}{{\r{U}}}1
	{Ý}{{\'Y}}1
	{Ž}{{\v{Z}}}1
}
\lstset{language=C}


% údaje na titulní straně
\title{Virtuální souborový systém\\na bázi NTFS}
\def \thesubtitle {Semestrální práce z předmětu KIV/ZOS}
\author{Patrik Harag}
\def \theauthoremail {harag@students.zcu.cz}
\def \theauthorid {(A15B0034P)}

\begin{document}

\begin{titlepage}
	\begin{figure}
		\includegraphics[height=50mm]{img-fav-logo}
	\end{figure}
	
	\centering
	{\large \hspace{1mm} \par} % tady musí být nějaký text jinak nefunguje vertikální odsazení
	\vspace{15ex}
	
	{\scshape\Large \thesubtitle \par}
	\vspace{1.5ex}
	{\huge\bfseries \thetitle \par}
	\vspace{2ex}
	{\Large\itshape \theauthor \par}
	\vspace{2ex}
	{\texttt{\theauthoremail} \par}
	\vspace{1ex}
	{\texttt{\theauthorid} \par}
	\vspace{5ex}
	%{{Celková doba řešení: \textgreaterX h} \par}
	
	\vfill

	{\large \today\par}
\end{titlepage}

% strana s obsahem
\setcounter{page}{0} 
\tableofcontents
\thispagestyle{empty}


\chapter{Úvod}
\section{Zadání}
Implementace virtuálního souborového systému na bázi NTFS s podporou následujících příkazů:
\begin{itemize}
	\item \verb|pwd| -- vypíše aktuální cestu.
	\item \verb|cd <path>| -- změní aktuální cestu.
	\item \verb|cp <src_path> <dest_path>| -- zkopíruje soubor.
	\item \verb|mv <src_path> <dest_path>| -- přesune soubor nebo složku.
	\item \verb|rm <path>| -- smaže soubor.
	\item \verb|mkdir <path>| -- vytvoří adresář.
	\item \verb|rmdir <path>| -- smaže prázdný adresář.
	\item \verb|ls [path]| -- vypíše obsah adresáře.
	\item \verb|cat <path>| -- vypíše obsah souboru.
	\item \verb|info <path>| -- vypíše informace o souboru/adresáři
	\item \verb|incp <src_path> <dest_path>| -- nahraje soubor z virtuálního \emph{FS} do hostitelského \emph{FS}.
	\item \verb|outcp <src_path> <dest_path>| -- nahraje soubor z hostitelského \emph{FS} do virtuálního \emph{FS}.
	\item \verb|load <path>| -- načte soubor s příkazy a sekvenčně je vykoná.
\end{itemize}

\noindent
Dále implementace defragmentace a kontroly konzistence.\\

\noindent
Více v přiloženém zadání.


\chapter{Programátorská dokumentace}

Program byl vyvinut v C 99.

\section{Seznam zdrojových souborů}
\begin{itemize}
	\item \emph{main.c} -- obsahuje vstupní funkci, zpracovává parametry a spouští REPL.
	\item \emph{utils.c/h} -- obsahuje různé pomocné funkce.
	\item \emph{repl.c/h} -- obsahuje \emph{read–eval–print-loop} (CLI).
	\item \emph{commands.c/h} -- obsahuje obsluhu všech příkazů, především ošetření vstupů.
	\item \emph{io.c/h} -- řeší zápis/čtení pseudo NTFS do/ze souboru.
	\item \emph{vfs.c/h} -- obsahuje základní funkce pro práci s \emph{VFS} (Virtual File System).
	\item \emph{vfs-bitmap.c/h} -- obsahuje nízkoúrovňové funkce pro práci s bitmapou.
	\item \emph{vfs-boot-record.c/h} -- obsahuje nízkoúrovňové funkce pro práci s \emph{boot record}.
	\item \emph{vfs-mft.c/h} -- obsahuje nízkoúrovňové funkce pro práci s \emph{MFT}.
	\item \emph{vfs-module-allocation.c/h} --  obsahuje funkce, které umožňují alokovat a uvolňovat různé zdroje.
	\item \emph{vfs-module-consistency.c/h} -- obsahuje funkce pro ověřování konzistence \emph{VFS}.
	\item \emph{vfs-module-defragmentation.c/h} -- obsahuje funkce pro defragmentaci.
	\item \emph{vfs-module-find.c/h} -- obsahuje funkce pro vyhledávání \emph{MFT} položek.
	\item \emph{vfs-module-utils.c/h} -- obsahuje různé pomocné funkce pro práci s \emph{VFS}.
	\item \emph{vfs-module-validation.c/h} -- obsahuje různé validační funkce.
\end{itemize}

\section{Důležité datové struktury}

\subsubsection*{VFS}
Tato datová struktura, která se nachází v souboru \emph{vfs.h}, uchovává všechny důležité informace o \emph{VFS}.

\begin{lstlisting}
typedef struct _VFS {
	char* file_name;
	FILE* file;
	bool initialized;
	
	BootRecord* boot_record;
	Mft* mft;
	Bitmap* bitmap;
} VFS;
\end{lstlisting}

\subsubsection*{BootRecord}
Struktura ze souboru \emph{vfs-boot-record.h}, která definuje \emph{boot record} tak, jak se bude ukládat do souboru.

\begin{lstlisting}
typedef struct _BootRecord {
	char signature[16];              // login autora FS
	char volume_descriptor[48];      // popis vygenerovaného FS
	int32_t disk_size;               // maximální velikost obsahu
	int32_t cluster_size;            // velikost clusteru
	int32_t cluster_count;           // pocet clusteru
	int32_t mft_item_count;          // počet položek v mft
	int32_t bitmap_size;             // velikost bitmapy v bytech
} BootRecord;
\end{lstlisting}

\subsubsection*{Bitmap}
Struktura ze souboru \emph{vfs-bitmap.h}, která definuje strukturu ukládající obsah bitmapy.

\begin{lstlisting}
typedef struct _Bitmap {
	int32_t length;
	unsigned char* data;
	int32_t cluster_count;
} Bitmap;
\end{lstlisting}

\newpage
\subsubsection*{Mft, MftItem, MftFragment}
V souboru \emph{vfs-mft.h} jsou definovány struktury pro práci s \emph{MFT}.

\begin{lstlisting}
/** Struktura fragmentu MFT položky tak, jak se bude ukládat do souboru */
typedef struct _MftFragment {
	int32_t start_index;     // start adresa
	int32_t count;           // pocet clusteru ve fragmentu
} MftFragment;

/** Struktura MFT položky tak, jak se bude ukládat do souboru */
typedef struct _MftItem {
	int32_t uid;             // UID polozky, pokud UID = UID_ITEM_FREE, je polozka volna
	int32_t parent_uid;
	int8_t is_directory;     // soubor, nebo adresar
	int8_t item_order;       // poradi v MFT pri vice souborech, jinak 1
	int8_t item_total;       // celkovy pocet polozek v MFT
	char item_name[VFS_MFT_ITEM_NAME_LENGTH];
	int32_t item_size;       // velikost souboru v bytech
	MftFragment fragments[VFS_MFT_FRAGMENTS_COUNT]; //fragmenty souboru
} MftItem;

/** Pomocná struktura pro MFT */
typedef struct _MFT {
	int32_t length;
	MftItem* data;  // pole struktur
} Mft;
\end{lstlisting}


\section{Defragmentace}
Defragmentace může být vyvolána příkazem \verb|defragment|. 

Je implementována úplná defragmentace disku. Po provedení úplné defragmentace se každý soubor skládá z maximálně jednoho fragmentu a všechny využité clustery jsou umístěny na začátek disku.

\subsection*{Implementace}
Algoritmus funguje tak, že postupně, od prvního, procházíme clustery a na dané místo se snažíme umístit obsah tak, aby po dokončení tohoto cyklu byl disk defragmentován.

Při zpracování dalšího clusteru vždy najdeme první následující neprázdný cluster a podle něj určíme MFT položku (pomocí předpřipravené tabulky), která jej využívá. Dalším cílem bude tuto MFT položku, tedy jejích \emph{n} clusterů umístit na daný cluster a případně \emph{n - 1} následujících clusterů.

Ve vnitřním cyklu postupně kopírujeme clustery a v případě, že některý cílový cluster není volný, realokujeme celý soubor, kterému patří někam pryč (pokud je dostatek místa, tak někam dál, abychom jej hned v příští iteraci nemuseli opět realokovat). Po skončení vnitřního cyklu pokračujeme za posledním umístěným clusterem.

\section{Kontrola konzistence}
Kontrola konzistence je prováděna při každém načtení virtuálního souborového systému ze souboru, případně může být vyvolána příkazem \verb|check consistency|.\\

\noindent
Jsou prováděny dvě kontroly:
\begin{itemize}
	\item Zda velikosti položek odpovídají počtu jejich clusterů.
	\item Zda není cluster referencován více položkami.
\end{itemize}

\subsection*{Implementace}
Úloha je paralelizována. Je vytvořeno \emph{n} vláken a každé vlákno si říká o další položky ke kontrole. Nad položkou provede obě dvě kontroly.\\

\noindent
Je přitom využívána sdílená struktura společná pro všechny vlákna:
\begin{lstlisting}
typedef struct _CheckContext {
	VFS* vfs;
	int32_t i;  // index další položky ke zpracování
	bool consistent;
	pthread_mutex_t mutex;
	MftItem** clusters;  // tabulka cluster index => MftItem*
} CheckContext;
\end{lstlisting}


\chapter{Uživatelská dokumentace}

\paragraph{Sestavení}
Pro sestavení je vyžadován GNU Linux.
Sestavení proběhne po zadání příkazu \verb|make| v kořenovém adresáři.

\paragraph{Spuštění}
Pro spuštění je vyžadován GNU Linux.
Spuštění proběhne po zadání příkazu \verb|./build/ntfs-dist <název souboru>|.

\section{Seznam příkazů}

\paragraph{Inicializační příkazy}
\begin{itemize}
	\item \verb|init| -- inicializuje \emph{VFS} o velikosti disku 10 kB (jako \verb|init 10240|). Velikost clusteru je 256 B.
	\item \verb|init <disk_size_in_bytes>| -- inicializuje \emph{VFS}. K zadané velkosti se připočte ještě \emph{boot record}, bitmapa a \emph{MFT} (které bude mít cca 10\% velikosti disku). Velikost clusteru je 256 B.
	\item \verb|init <cluster_count> <mft_items_count>| -- inicializuje \emph{VFS}. Velikost clusteru je 256 B.
	\item \verb|init <cluster_count> <cluster_size_in_bytes> <mft_items_count>| -- inicializuje \emph{VFS}.
\end{itemize}

\paragraph{Informativní a debug příkazy}
\begin{itemize}
	\item \verb|df| -- vypíše informace o obsazenosti.
	\item \verb|show parameters| -- vypíše parametry \emph{VFS}.
	\item \verb|show bitmap| -- zbrazí bitmapu (0 $\rightarrow$ volný cluster, 1 $\rightarrow$ obsazený cluster).
	\item \verb|show mft| -- vypíše položky \emph{MFT}.
\end{itemize}

\paragraph{Základní příkazy}
\begin{itemize}
	\item \verb|pwd| -- vypíše aktuální cestu.
	\item \verb|cd <path>| -- změní aktuální cestu.
	\item \verb|cp <src_path> <dest_path>| -- zkopíruje soubor.
	\item \verb|mv <src_path> <dest_path>| -- přesune soubor nebo složku.
	\item \verb|rm <path>| -- smaže soubor.
	\item \verb|mkdir <path>| -- vytvoří adresář.
	\item \verb|rmdir <path>| -- smaže prázdný adresář.
	\item \verb|ls [path]| -- vypíše obsah adresáře.
	\item \verb|cat <path>| -- vypíše obsah souboru.
	\item \verb|info <path>| -- vypíše informace o souboru/adresáři
	\item \verb|incp <src_path> <dest_path>| -- nahraje soubor z virtuálního \emph{FS} do hostitelského \emph{FS}.
	\item \verb|outcp <src_path> <dest_path>| -- nahraje soubor z hostitelského \emph{FS} do virtuálního \emph{FS}.
	\item \verb|load <path>| -- načte soubor s příkazy a sekvenčně je vykoná.
\end{itemize}

\paragraph{Ostatní příkazy}
\begin{itemize}
	\item \verb|reallocate <path>| -- realokuje soubor na jinou pozici - to ho může defragmentovat, pokud je k dispozici vhodné místo.
	\item \verb|defragment| -- provede úplnou defragmentaci disku.
	\item \verb|check consistency| -- zkontroluje konzistenci (kontrola konzistence se automaticky spouští po načtení).
	\item \verb|disable first-fit| -- deaktivuje metodu pro vyhledávání volného místa na disku \emph{first fit}. Po deaktivaci budou vybírány vždy první volné clustery, bude tak docházet k větší fragmentaci. Slouží pro testovací účely.
\end{itemize}


\chapter{Závěr}
Implementoval jsem virtuální souborový systém na bázi NTFS podle zadání. Jsou podporovány všechny požadované příkazy, defragmentace a kontrola konzistence. Byly také vytvořeno několik příkazů navíc, většina z nich pro účely kontroly správného fungování programu.

Během celého vývoje jsem vytvářel funkční testy, což se mi velmi vyplatilo především v pozdějších fázích vývoje, kdy už by byl problém \uv{ohlídat} takové množství funkcí a vlastností. Zvláště v jazyce, jako je ANSI C. 

\end{document}